\chapter{Исследовательская часть}

\section{Технические характеристики}

Технические характеристики устройства, на котором выполнялось исследование:
\begin{itemize}
  \item операционная система EndeavourOS 64бит;
  \item версия ядра Linux 6.12.3-arch1-1;
  \item 13th Gen Intel(R) Core(TM) i5-13500H 4.70 ГГц 12 ядер (4
  производительных + 8 энергоэффективных) 16 логических процессоров;
  \item оперативная память 16ГБ с частотой 5200МГц.
\end{itemize}

\section{Описание исследования}

Было проведено несколько исследований.

\subsection*{Исследование времени генерации сцены}

Было проведено исследование зависимости времени генерации сцены от количества строк/столбцов матрицы сцены. В ходе исследования производилась генерация только квадратных сцен, то есть количество строк и столбцов в матрице совпадало.

Для каждого размера матрицы замер времени проводился 10 раз. В качестве результата сохранялось среднее значение потраченного времени.

Для замеров процессорного времени использовалась библиотека chrono, являющаяся частью стандартной библиотеки языка c++~\cite{cpp}.

Результаты временных замеров представлены в виде графика на рисунке~\ref{fig:qwfc_graph}.

\begin{figure}[h!]
  
\begin{tikzpicture}
  \begin{axis}[
    xlabel={Количество столбцов/строк в матрице сцены},
    ylabel={Время вычисления (мс)},
    xmin=0, xmax=205,
    ymin=-50, ymax=1650,
    xtick={0,25,50,75,100,125,150,175,200},
    ytick={0,200,400,600,800,1000,1200,1400,1600},
    legend pos=north west,
    ymajorgrids=true,
    grid style=dashed,
    ]
    
    \addplot[
      color=blue,
      mark=square,
    ]
    coordinates {
(10,	348.793/1000)(15,	824.85/1000)(20,	1519.78/1000)(25,	2564.9/1000)(30,	3889.52/1000)(35,	5513.14/1000)(40,	7619.98/1000)(45,	10776.5/1000)(50,	14038.7/1000)(55,	19070.3/1000)(60,	23430.7/1000)(65,	28802.4/1000)(70,	36098.5/1000)(75,	41871.9/1000)(80,	51021/1000)(85,	61215.6/1000)(90,	72200.7/1000)(95,	83964/1000)(100,	98533/1000)(105,	113585/1000)(110,	133330/1000)(115,	151468/1000)(120,	179679/1000)(125,	203223/1000)(130,	243183/1000)(135,	270402/1000)(140,	316742/1000)(145,	365271/1000)(150,	428129/1000)(155,	488139/1000)(160,	569665/1000)(165,	648947/1000)(170,	737800/1000)(175,	861445/1000)(180,	952674/1000)(185,	1.07038e+06/1000)(190,	1.20905e+06/1000)(195,	1.42681e+06/1000)(200,	1.58427e+06/1000)
      };
  \end{axis}
\end{tikzpicture}
\caption{График зависимости времени генерации сцены от количества строк/столбцов матрицы сцены}
\label{fig:qwfc_graph}
\end{figure}
    

\subsection*{Исследование времени создания карты теней}

Было проведено исследование зависимости времени создания карты теней от размеров матрицы глубин карты теней. Размеры матрицы по оси $Ox$ и $Oy$ в процессе исследования совпадали, то есть матрица глубин была квадратной.

Количество видимых граней на сцене во время замеров оставалось постоянным.

Для замеров процессорного времени использовалась библиотека chrono, являющаяся частью стандартной библиотеки языка c++~\cite{cpp}.

Результаты временных замеров представлены в виде графика на рисунке~\ref{fig:smap_graph}.

\begin{figure}[h!]
  \begin{tikzpicture}
    \begin{axis}[
      xlabel={Количество строк/столбцов в матрице глубин карты теней},
      ylabel={Время вычисления (мс)},
      xmode=log,
      ymode=log,
      log basis y={2},
      log basis x={2},
      xmin=400, xmax=17000,
      ymin=1.5, ymax=1650,
      ticklabel style={
        /pgf/number format/fixed,
        /pgf/number format/fixed zerofill,
        /pgf/number format/precision=0,
        /pgf/number format/1000 sep={},
        },
      xticklabel={\pgfmathparse{2^(\tick)}\pgfmathprintnumber{\pgfmathresult}},
      log ticks with fixed point,
      scaled ticks=false,
      xtick={512,1024,2048,4096,8192,16384},
      ytick={1.79,6.18,21.43,78.83,307.33,1466.7},
      legend pos=north west,
      ymajorgrids=true,
      grid style=dashed,
      ]
      
      \addplot[
        color=blue,
        mark=square,
      ]
      coordinates {
        (512, 1.79759e+06 / 1000 / 1000)
        (1024, 6.181e+06 / 1000 / 1000)
        (2048, 2.1439e+07 / 1000 / 1000)
        (4096, 7.88313e+07 / 1000 / 1000)
        (8192, 3.0733e+08 / 1000 / 1000)
        (16384, 1.4766e+09 / 1000 / 1000)
        };
    \end{axis}
  \end{tikzpicture}
  \caption{График зависимости времени заполнения квадратной матрицы глубин карты теней от её размеров (кол-ва элементов в её строках/столбцах)}
  \label{fig:smap_graph}
\end{figure}

Для этого графика были использованы логарифмические шкалы осей $Ox$ и $Oy$.

\subsection*{Исследование времени генерации кадра}

Было проведено исследование зависимости времени генерации от количества видимых граней на сцене.

Разрешение экрана (то есть размеры матрицы цветов кадра) в процессе исследования оставались постоянными.

Для замеров процессорного времени использовалась библиотека chrono, являющаяся частью стандартной библиотеки языка c++~\cite{cpp}.

Результаты временных замеров представлены в виде графика на рисунке~\ref{fig:render_graph}.

\begin{figure}[h!]
  \begin{tikzpicture}
    \begin{axis}[
      xlabel={Количество видимых граней},
      ylabel={Время генерации кадра (мс)},
      xmode=log,
      ymode=log,
      xmin=8, xmax=100000,
      ymin=0, ymax=35,
      ticklabel style={
        /pgf/number format/fixed,
        /pgf/number format/fixed zerofill,
        /pgf/number format/precision=5,
        /pgf/number format/1000 sep={},
        },
      xtick={0,10,60,250,800,3200,20000,75000},
      ytick={0.75, 1.07, 1.97, 3.19, 6.08, 11.69, 18.92, 25},
      log ticks with fixed point,
      legend pos=north west,
      ymajorgrids=true,
      grid style=dashed,
      ]
      
      \addplot[
        color=blue,
        mark=square,
      ]
      coordinates {(10, 756071/1000/1000)(60, 1.07985e+06/1000/1000)(250, 1.97406e+06/1000/1000)(800, 3.19032e+06/1000/1000)(3200, 6.08609e+06/1000/1000)(20000, 1.16905e+07/1000/1000)(43000, 1.89214e+07/1000/1000)(75000, 2.57847e+07/1000/1000)
        };
    \end{axis}
  \end{tikzpicture}
  \caption{График зависимости времени генерации кадра от количества видимых граней}
  \label{fig:render_graph}
\end{figure}

Для этого графика были использованы логарифмические шкалы осей $Ox$ и $Oy$.

\newpage

\section{Вывод}

По результатам исследования обнаружены следующие зависимости:
\begin{itemize}
  \item зависимость времени генерации квадратной сцены от количества строк/столбцов матрицы близка к степенной функции; 
  \item зависимость времени создания квадратной карты теней от количества строк/столбцов её матрицы глубины при одинаковом количестве видимых граней близка к линейной;
  \item зависимость времени генерации кадра от количества объектов на сцене при одинаковом разрешении экрана близка к линейной.
\end{itemize}
