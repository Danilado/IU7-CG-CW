\chapter{Исследовательская часть}

\section{Технические характеристики}

Технические характеристики устройства, на котором выполнялось исследование:
\begin{itemize}
  \item операционная система EndeavourOS 64бит;
  \item версия ядра Linux Linux 6.12.3-arch1-1;
  \item 13th Gen Intel(R) Core(TM) i5-13500H 4.70 ГГц 12 ядер (4
  производительных + 8 энергоэффективных) 16 логических процессоров;
  \item оперативная память 16ГБ с частотой 5200МГц.
\end{itemize}

\section{Описание исследования}

Было произведено несколько исследований.

\subsection*{Исследование времени генерации сцены}

Было проведено исследования времени генерации сцены от размеров матрицы. В ходе исследования производилась генерация квадратных сцен.

Для каждого размера матрицы, замер времени проводился 10 раз. В качестве результата сохранялось среднее значение затраченного времени.

Для замеров времени использовалась библиотека chrono, являющаяся частью стандартной библиотеки языка c++~\cite{cpp}.

Результаты временных замеров представлены в виде графика на рисунке~\ref{fig:qwfc_graph}.

\begin{figure}[h!]
  \centering
  \includesvg[width=.7\textwidth]{qwfcprof.svg}
  \caption{График зависимости времени генерации сцены от размеров сцены}
  \label{fig:qwfc_graph}
\end{figure}

\subsection*{Исследование времени создания карты теней}

Было проведено исследования зависимости времени создания карты теней от размеров буфера глубины. Размеры буфера по оси $Ox$ и $Oy$ в процессе исследования совпадали, то есть буфер был квадратным.

Количество моделей на сцене во время временных замеров оставалось постоянным.

Для замеров времени использовалась библиотека chrono, являющаяся частью стандартной библиотеки языка c++~\cite{cpp}.

Результаты временных замеров представлены в виде графика на рисунке~\ref{fig:smap_graph}.

\begin{figure}[h!]
  \centering
  \includesvg[width=.7\textwidth]{smapprof.svg}
  \caption{График зависимости времени создания карты теней от её разрешения}
  \label{fig:smap_graph}
\end{figure}

\subsection*{Исследование времени генерации кадра}

Было проведено исследования зависимости времени генерации от размеров матрицы сцены. В процессе исследования рассматривались только квадратные сцены.

Разрешение экрана (то есть размеры буфера кадра) в процессе исследования оставались постоянными.

Для замеров времени использовалась библиотека chrono, являющаяся частью стандартной библиотеки языка c++~\cite{cpp}.

Результаты временных замеров представлены в виде графика на рисунке~\ref{fig:render_graph}.

\begin{figure}[h!]
  \centering
  \includesvg[width=.7\textwidth]{rendprof.svg}
  \caption{График зависимости времени генерации кадра от размеров сцены}
  \label{fig:render_graph}
\end{figure}

\newpage

\section{Вывод}

Проанализировав графики, можно увидеть, что:
\begin{itemize}
  \item зависимость времени генерации квадратной сцены от размера стороны матрицы близка к степенной функции; 
  \item зависимость времени создания квадратной карты теней от размера стороны её буфера глубины близка к степенной функции; 
  \item зависимость времени генерации кадра от размера сцены (а соответственно и количества объектов на ней) близка к линейной.
\end{itemize}
