\subsection{Удаление невидимых поверхностей}

Удаление невидимых поверхностей для этой работы будет проводиться по двум критериям:

\begin{itemize}
    \item скалярное произведение нормали поверхности и вектора направления камеры;
    \item попадание поверхности в пирамиду видимости камеры.
\end{itemize}

Первый критерий позволяет удалить нелицевые (с точки зрения камеры) поверхности. Это позволяет существенно уменьшить количество поверхностей, которые придётся обрабатывать при генерации кадра.

Для определения видимости поверхности, достаточно вычислить значения скалярного произведения $(\vec{N}, \vec{V})$, где $\vec{N}$ --- вектор нормали поверхности, а $\vec{V}$ --- вектор направления камеры. Тогда, при значении $(\vec{N}, \vec{V})$ меньше нуля, поверхность будет лицевой, а во всех остальных случаях --- нелицевой.

Следует отметить, что для правильного определения нелицевых граней, их необходимо также переводить в пространство перспективной проекции, если оно используется при построении сцены.

Второй критерий позволяет удалить те поверхности, которые не попадают в поле зрения камеры. Часто для упрощения вычислений этого критерия, используется не сама поверхность (которая может задаваться сложной фигурой), а параллелепипед, вписывающий данную поверхность в себя.

Удобно использовать такой параллелепипед, стороны которого параллельны осям системы координат камеры. Тогда, для вычисления его параметров будет достаточно вычислить максимальные и минимальные значения каждой из координат.

После этого, выполняется определение видимости этого параллелепипеда путём пересечения его с пирамидой видимости камеры~\cite{Rogers}.

Определение пересечения бесконечной пирамиды и прямоугольника тривиально, и сводится к определению определению пересечения прямых.