\subsection{Матрицы преобразования}

В данном алгоритме (и структурах данных) под словосочетанием ``матрица преобразования'' подразумеваются такая матрица, при умножении всех точек объекта на которую, будет получен преобразованный объект~\cite{CGPaP}.

Данная матрица, как правило, получается как результат произведения нескольких матриц аффинных преобразований.

В данной работе используются следующие аффинные преобразования:

\begin{itemize}
    \item перенос;
    \item масштабирование;
    \item поворот.
\end{itemize}

Матрица переноса представляется следующим образом:
$$
\begin{pmatrix}
    1 & 0 & 0 & 0 \\
    0 & 1 & 0 & 0 \\
    0 & 0 & 1 & 0 \\
    dx & dy & dz & 1
\end{pmatrix}
$$

здесь, $dx, dy, dz$ --- это смещения по осям $Ox, Oy$ и $Oz$ соответственно

Матрица масштабирования представляется следующим образом:

$$
\begin{pmatrix}
    kx & 0 & 0 & 0 \\
    0 & ky & 0 & 0 \\
    0 & 0 & kz & 0 \\
    0 & 0 & 0 & 1
\end{pmatrix}
$$

здесь, $kx, ky, kz$ --- это коэффициенты масштабирования по осям $Ox, Oy$ и $Oz$ соответственно

Матрицы поворота представляются по-разному для каждой оси.

Матрица поворота вокруг оси $Ox$ представляется следующим образом:

$$
\begin{pmatrix}
    1 & 0 & 0 & 0 \\
    0 & cos \alpha & sin \alpha & 0 \\
    0 & -sin \alpha & cos \alpha & 0 \\
    0 & 0 & 0 & 1
\end{pmatrix}
$$

Матрица поворота вокруг оси $Oy$ представляется следующим образом:

$$
\begin{pmatrix}
    cos \alpha & 0 & -sin \alpha & 0 \\
    0 & 1 & 0 & 0 \\
    sin \alpha & 0 & cos \alpha & 0 \\
    0 & 0 & 0 & 1
\end{pmatrix}
$$

Матрица поворота вокруг оси $Oz$ представляется следующим образом:

$$
\begin{pmatrix}
    cos \alpha & sin \alpha & 0 & 0 \\
    -sin \alpha & cos \alpha & 0 & 0 \\
    0 & 0 & 1 & 0 \\
    0 & 0 & 0 & 1
\end{pmatrix}
$$

Во всех матрицах вращения, $\alpha$ --- угол поворота вокруг соответствующей оси в радианах.