\subsection{Приведение к пространству камеры}

Поскольку положение и направление камеры описывается матрицей преобразований, приведение точки к пространству координат камеры равносильно применению к данной точке обратных преобразований, применённых к камере.

Например, при смещении камеры на 10 точек по оси $Ox$, относительно камеры объекты сцены сместятся на -10 точек по той же оси. При повороте камеры на 10 градусов вокруг оси $Ox$, объекты сцены повернутся на -10 градусов вокруг той же оси относительно координат камеры.

Пользуясь этим свойством, для упрощения перевода координат точек в пространство камеры, можно воспользоваться следующим приёмом: при преобразовании камеры (координат и направления) можно записывать в её матрицу преобразования обратные значения.

Таким образом, для приведения точки к пространству камеры, достаточно применить к этой точке матрицу преобразования камеры.