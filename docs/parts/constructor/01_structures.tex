\section{Описание структур данных}

\subsection*{Структуры данных, описывающие содержимое сцены}

\begin{enumerate}
    \item сцена представляет собой список, содержащий указатели на объекты сцены (дома, дороги и т.п.);
    \item объекты сцены включают в себя следующие данные:\begin{itemize}
        \item массив поверхностей;
        \item матрица преобразований.
    \end{itemize}
    \item поверхность включает в себя следующие данные: \begin{itemize}
        \item массив вершин;
        \item вектор нормали;
        \item спектральные характеристики.
    \end{itemize}
    \item вершины включают в себя следующие данные: \begin{itemize}
        \item координаты вершины.
    \end{itemize}
    \item источник света включает в себя следующие данные: \begin{itemize}
        \item координаты источника.
    \end{itemize}
    \item камера включает в себя следующие данные: \begin{itemize}
        \item координаты камеры; 
        \item матрица преобразований направления камеры.
    \end{itemize}
\end{enumerate}

Следует отметить, что матрица преобразований камеры описывает как её координаты, так и её направление. 

Камера с единичной матрицей трансформации находится в точке с координатами $(0,0,0)$ и направлена вдоль оси $X$.

\subsection*{Структуры данных, необходимые для генерации сцены}

В процессе генерации, содержимое сцены описывается матрицей (двумерным массивом), каждая ячейка которой означает нахождение определённого объекта на координатах, соответствующих данной ячейке матрицы.

Объекты матрицы содержат следующие данные:
\begin{itemize}
    \item идентификатор типа объекта;
    \item таблица возможных идентификаторов типов соседних клеток (по одной записи на каждую соседнюю ячейку).
\end{itemize}
