\section{Описание структур данных}

\subsection*{Структуры данных, описывающие содержимое сцены}

\begin{enumerate}
    \item Сцена представляет собой список, содержащий указатели на объекты сцены (дома, дороги и т.п.);
    \item Объекты сцены включают в себя следующие данные:\begin{itemize}
        \item массив граней;
        \item матрица преобразований.
    \end{itemize}
    \item Грань включает в себя следующие данные: \begin{itemize}
        \item массив вершин;
        \item вектор нормали;
        \item цвет.
    \end{itemize}
    \item Вершины включают в себя следующие данные: \begin{itemize}
        \item координаты вершины.
    \end{itemize}
    \item Источник света включает в себя следующие данные: \begin{itemize}
        \item координаты источника.
    \end{itemize}
    \item Камера включает в себя следующие данные: \begin{itemize}
        \item координаты камеры; 
        \item углы поворота камеры вокруг осей $Ox$ и $Oy$.
    \end{itemize}
\end{enumerate}

\subsection*{Структуры данных, необходимые для генерации сцены}

В процессе генерации, содержимое сцены описывается матрицей (двумерным массивом), каждая ячейка которой может принимать одно из предопределённых состояний. Состояние ячейки отображает тип объекта, находящегося на координатах, соответствующих данной ячейке матрицы.

Состояние ячейки матрицы содержит следующие данные:
\begin{itemize}
    \item тип объекта, соответствующий данному состоянию;
    \item 4 массива возможных состояний соседних ячеек (по одному массиву на каждую соседнюю ячейку).
\end{itemize}
