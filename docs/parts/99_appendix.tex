\renewcommand\thechapter{A}

\ssr{ПРИЛОЖЕНИЕ A}

\begin{lstlisting}[caption={Использование RenderVisitor на сцену}, label={lst:render_scene}]
void RenderVisitor::visit(Scene &ref) {
  ShadowMap &smap = Singleton<ShadowMap>::instance();
  auto resolution = ctx->getResolution();
  colors.resize(resolution.second, std::vector<Color>(resolution.first));
  depth.resize(resolution.second,
               std::vector<double>(resolution.first,
                                   std::numeric_limits<double>::infinity()));
  pixmaps.clear();

  tbb::parallel_for_each(ref.begin(), ref.end(), [this](const auto &newref) {
    newref.second->accept(*this);
  });

  for (const MyPixmap &pmap :
       pixmaps | std::views::filter([](const MyPixmap &pmap) {
         return !pmap.getDepth().empty() && !pmap.getDepth().front().empty();
       })) {
    auto &pdepth = pmap.getDepth(); auto offset = pmap.getOffset();
    auto &pcolor = pmap.getColor(); auto &pbright = pmap.getBrightness();

    for (int x = 0; x < pdepth.front().size(); ++x)
      for (int y = 0; y < pdepth.size(); ++y) {
        if (pdepth[y][x] < depth[y + offset.second][x + offset.first]) {
          depth[y + offset.second][x + offset.first] = pdepth[y][x];
          if (smap.isValid())
            colors[y + offset.second][x + offset.first] =
                pcolor * pbright[y][x];
          else
            colors[y + offset.second][x + offset.first] = pcolor;
        }
      }
  }

  ctx->setImage(colors);
}
\end{lstlisting}

\begin{lstlisting}[caption={Использование RenderVisitor на модель}, label={lst:render_model}]
void RenderVisitor::visit(FaceModel &ref) {
  ShadowMap &smap = Singleton<ShadowMap>::instance();

  std::shared_ptr<TransformationMatrix> transf = ref.getTransformation();

  CameraManager &cm = Singleton<CameraManager>::instance();

  std::shared_ptr<BaseCamera> cam =
      std::dynamic_pointer_cast<BaseCamera>(cm.getCamera());
  std::shared_ptr<TransformationMatrix> camtransf = cam->getTransformation();

  PTSCAdapter->setCamera(cm.getCamera());
  Point3D cam_position = cam->getPosition();
  auto resolution = ctx->getResolution();

  auto range =
      ref.getFaces() |
    std::views::filter(std::bind(&Face::isVisible, std::placeholders::_1,
                                    cam_position, *transf)) |
      std::views::filter([&transf, &camtransf](const Face &f) {
        return std::ranges::any_of(f.getPoints(), [&transf, &camtransf](
                                                      const Point3D &pt) {
    return camtransf->apply(transf->apply(pt)).z >= 2 * MyMath::EPS;
        });
      });

  tbb::parallel_for_each(
      range.begin(), range.end(),
      [this, &camtransf, &transf, &resolution, &smap](const Face &face) {
        pixmaps.emplace_back(face.getPixmap(
            [transf](const Point3D &pt) { return transf->apply(pt); },
            [&camtransf](const Point3D &pt) { return camtransf->apply(pt); },
            [this, &smap](const Point3D &pt) {
              return PTSCAdapter->convert(pt);
            },
            resolution.first, resolution.second, smap, true));
      });
}  
\end{lstlisting}

\iffalse
\begin{lstlisting}[caption={Растеризация треугольника --- начало}, label={lst:render_triangle}]
MyPixmap Triangle::getPixmap(
    std::function<Point3D(const Point3D &)> to_world,
    std::function<Point3D(const Point3D &)> to_camera,
    std::function<std::shared_ptr<Point2D>(const Point3D &)> to_screen,
    size_t screen_width, size_t screen_height, const Color &color,
    const ShadowMap &smap, bool center_axis) const {
  int centerX = center_axis ? static_cast<int>(screen_width / 2) : 0;
  int centerY = center_axis ? static_cast<int>(screen_height / 2) : 0;

  Point3D t0 = to_world(p1); 
  Point3D t1 = to_world(p2); 
  Point3D t2 = to_world(p3);

  Point3D v0 = to_camera(t0); 
  Point3D v1 = to_camera(t1); 
  Point3D v2 = to_camera(t2);

  std::shared_ptr<Point2D> pt1_screen = to_screen(v0);
  std::shared_ptr<Point2D> pt2_screen = to_screen(v1);
  std::shared_ptr<Point2D> pt3_screen = to_screen(v2);

  Point2D p0_2d(pt1_screen->x + centerX, pt1_screen->y + centerY);
  Point2D p1_2d(pt2_screen->x + centerX, pt2_screen->y + centerY);
  Point2D p2_2d(pt3_screen->x + centerX, pt3_screen->y + centerY);

  if (p1_2d.y < p0_2d.y) {
    std::swap(p0_2d, p1_2d); std::swap(v0, v1); std::swap(t0, t1);
  }
  if (p2_2d.y < p1_2d.y) {
    std::swap(p1_2d, p2_2d); std::swap(v1, v2); std::swap(t1, t2);
  }
  if (p1_2d.y < p0_2d.y) {
    std::swap(p0_2d, p1_2d); std::swap(v0, v1); std::swap(t0, t1);
  }

  auto interpolate = [](double y1, double x1, double y2, double x2, double y) {
    if (y1 == y2) return x1;
    return x1 + (x2 - x1) * (y - y1) / (y2 - y1);
  };
\end{lstlisting}
\begin{lstlisting}[caption={Продолжение Растеризация треугольника}, label={lst:render_triangle1}]
  auto interpolateZ = [](double y1, double z1, double y2, double z2, double y) {
    if (y1 == y2) return z1;
    return z1 + (z2 - z1) * (y - y1) / (y2 - y1);
  };

  auto interpolatePart = [](double v1, double v2, int y1, int y2, int y) {
    return v1 + (v2 - v1) * (y - y1) / static_cast<double>(y2 - y1);
  };

  auto interpolatePoint = [&interpolatePart](const Point3D &p1,
                                             const Point3D &p2, int y1, int y2,
                                             int y) {
    return Point3D(interpolatePart(p1.x, p2.x, y1, y2, y),
                   interpolatePart(p1.y, p2.y, y1, y2, y),
                   interpolatePart(p1.z, p2.z, y1, y2, y));
  };

  int minY = std::max(0, static_cast<int>(std::floor(p0_2d.y)));
  int minYClamped = std::clamp(minY, 0, static_cast<int>(screen_height));
  int maxY = std::min(static_cast<int>(screen_height),
               static_cast<int>(std::ceil(p2_2d.y))) + 1; 
  int maxYClamped = std::clamp(maxY, 0, static_cast<int>(screen_height));

  int minX = std::min({static_cast<int>(p0_2d.x),
                       static_cast<int>(p1_2d.x),
                       static_cast<int>(p2_2d.x)});
  int minXClamped = std::clamp(minX, 0, static_cast<int>(screen_width));
  int maxX = std::max({static_cast<int>(p0_2d.x),
                static_cast<int>(p1_2d.x),
                static_cast<int>(p2_2d.x)}) + 1;
  int maxXClamped = std::clamp(maxX, 0, static_cast<int>(screen_width));

  if (maxXClamped - minXClamped <= 0 || maxYClamped - minYClamped <= 0)
    return MyPixmap(0, 0);

  MyPixmap res(maxXClamped - minXClamped, maxYClamped - minYClamped,
               minXClamped, minYClamped, smap.isValid());
\end{lstlisting}

\begin{lstlisting}[caption={Продолжение Растеризация треугольника}, label={lst:render_triangle2}]
  std::vector<std::vector<double>> &depth = res.getDepthE();
  std::vector<std::vector<double>> &brightness = res.getBrightnessE();
  res.setColor(color);

  for (int y = minYClamped; y < maxYClamped; ++y) {
    double xLeft, xRight, zLeft, zRight;
    Point3D worldLeft, worldRight;

    if (y < p1_2d.y) {
      xLeft = interpolate(p0_2d.y, p0_2d.x, p1_2d.y, p1_2d.x, y);
      xRight = interpolate(p0_2d.y, p0_2d.x, p2_2d.y, p2_2d.x, y);
      zLeft = interpolateZ(p0_2d.y, v0.z, p1_2d.y, v1.z, y);
      zRight = interpolateZ(p0_2d.y, v0.z, p2_2d.y, v2.z, y);

      worldLeft = interpolatePoint(t0, t1, p0_2d.y, p1_2d.y, y);
      worldRight = interpolatePoint(t0, t2, p0_2d.y, p2_2d.y, y);
    } else {
      xLeft = interpolate(p1_2d.y, p1_2d.x, p2_2d.y, p2_2d.x, y);
      xRight = interpolate(p0_2d.y, p0_2d.x, p2_2d.y, p2_2d.x, y);
      zLeft = interpolateZ(p1_2d.y, v1.z, p2_2d.y, v2.z, y);
      zRight = interpolateZ(p0_2d.y, v0.z, p2_2d.y, v2.z, y);

      worldLeft = interpolatePoint(t1, t2, p1_2d.y, p2_2d.y, y);
      worldRight = interpolatePoint(t0, t2, p0_2d.y, p2_2d.y, y);
    }

    if (xLeft > xRight) {
      std::swap(xLeft, xRight);
      std::swap(zLeft, zRight);
      std::swap(worldLeft, worldRight);
    }

int xStart = std::max(minXClamped, static_cast<int>(std::floor(xLeft)));
int xEnd = std::min(maxXClamped, static_cast<int>(std::ceil(xRight) +1.));

    if (xEnd - xStart <= 1)
      continue;
\end{lstlisting}
\newpage
\begin{lstlisting}[caption={Продолжение Растеризация треугольника}, label={lst:render_triangle3}]
    double z = zLeft;
    double dz = (xEnd >= xStart) ? (zRight - zLeft) / (xRight - xLeft) : 0;

    Point3D world = worldLeft;
    Point3D d_world = (xEnd > xStart)
                          ? (worldRight - worldLeft) / (xRight - xLeft)
                          : Point3D();

    if (std::ceil(xLeft) < xStart) {
      z += dz * (xStart - xLeft);
      world += d_world * (xStart - xLeft);
    }

    for (int x = xStart; x < xEnd; ++x) {
      depth[y - minYClamped][x - minXClamped] = z;
      if (smap.isValid()) {
        brightness[y - minYClamped][x - minXClamped] =
            smap.getBrightness(world);
      }
      z += dz;
      world += d_world;
    }
  }

  return res;
}
\end{lstlisting}
\fi

\begin{lstlisting}[caption={Метод фиксирования состояния ячейки класса QWFC}, label={lst:qwfcoll}]
void QWFC::collapseCell(int x, int y) {
  if (x < 0 || y < 0 || x >= width || y >= height)
    return;
  auto &possibleValues = matrix[y][x];
  if (possibleValues.size() == 1)
    return;

  double totalProbability = 0.0;
  for (const auto &value : possibleValues) {
    totalProbability += cells.at(value).getProbability();
  }

  if (totalProbability < MyMath::EPS) {
    throw std::runtime_error("Can not continue.");
  }

  double random = static_cast<double>(rand()) / static_cast<double>(RAND_MAX) *
                  totalProbability;
  double cumulativeProbability = 0.0;
  int chosenValue = *possibleValues.begin();

  for (const auto &value : possibleValues) {
    cumulativeProbability += cells.at(value).getProbability();
    if (random <= cumulativeProbability) {
      chosenValue = value;
      break;
    }
  }

  possibleValues.clear();
  possibleValues.insert(chosenValue);
}
\end{lstlisting}

% \begin{lstlisting}[caption={}, label={lst:}]
% \end{lstlisting}
\newpage

\renewcommand\thechapter{B}

\ssr{ПРИЛОЖЕНИЕ B}

Презентация состоит из N слайдов.