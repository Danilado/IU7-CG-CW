\section{Сравнение алгоритмов построения теней}

В данной работе будет использоваться метод теневых карт для построения теней на изображении.

\subsection*{Метод теневых карт}

Метод теневых карт основывается на построении карты теней методом заполнения Z-буфера с точки зрения источника света и сравнения этого буфера с точки зрения камеры для правильного затенения пикселей~\cite{Shadows}.

Процесс применения этого метода можно разбить на следующие этапы:

\begin{enumerate}
    \item создание теневых карт;
    \item сравнение глубин пикселей;    
    \item применение освещения.
\end{enumerate}

\subsubsection*{Создание теневых карт}

На этом этапе информация о сцене заносится в Z-буфер с точки зрения источника света. Вместо цвета и яркости пикселей в Z-буфер будет сохраняться значение глубины каждого фрагмента изображения. В результате такого заполнения, будет получена текстура, называемая теневой картой. Эта текстура будет использоваться для определения, находится ли фрагмент изображения в тени.

\subsubsection*{Сравнение глубин пикселей} 

На втором этапе информация о сцене заносится в Z-буфер с точки зрения камеры. Для каждого фрагмента изображения координаты преобразуются в координаты теневой карты и значение глубины фрагмента сравниваются  с соответствующим значением в теневой карте. 
Если значение глубины фрагмента больше, чем значение в теневой карте, фрагмент находится в тени.

\subsubsection*{Применение освещения} 

На последнем этапе полученная информация о тенях применяется к изображению: те пиксели (фрагменты) которые не находятся в тени, становятся светлее, а остальные - затеняются, в зависимости от интенсивности света.

\subsection*{Преимущества и недостатки}

Метод теневых карт в сочетании с алгоритмом Z-буфера имеет свои преимущества и недостатки. К преимуществам можно отнести:

\begin{itemize}
    \item высокую производительность для динамических сцен;
    \item возможность создания реалистичных теней для сложных объектов.
\end{itemize}

Однако существуют и недостатки:

\begin{itemize}
    \item при низком разрешении теневых карт, могут наблюдаться проблемы с качеством изображения;
    \item ограниченная точность теней для объектов: чем дальше расположен объект от источника света, тем менее точной будет его тень.
\end{itemize}

\subsection*{Вывод}

Метод теневых карт достаточен для достижения целей работы. Поскольку сцена имеет всего один источник света, не придётся производить большого количества вычислений при визуализации сцены, а качество теней должно быть достаточным, так как источник света расположен на условно бесконечном расстоянии от сцены, а следовательно равноудалён от всех объектов. 
