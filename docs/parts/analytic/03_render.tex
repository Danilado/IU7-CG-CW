\section{Сравнение алгоритмов удаления невидимых поверхностей}

Алгоритмы удаления невидимых линий и поверхностей используются в машинной графике для удаления тех объектов (или частей объектов), которые перекрываются другими. Задача удаления невидимых линий и поверхностей является одной из наиболее сложных в машинной графике~\cite{Rogers}.

В этом разделе будут рассмотрены несколько алгоритмов удаления невидимых поверхностей.

\subsection*{Алгоритм Варнока}

Алгоритм Варнока (или алгоритм художника) основывается на разбиении пространства на области, которые могут быть классифицированы, как сложные и простые.

Под сложной областью подразумевается такая область (часть пространства на экране), которая превышает по размеру один пиксель и в которой возникает любой из следующих случаев:
\begin{itemize}
    \item в область попали несколько объектов;
    \item в область попал один объект, который занимает не всю область.
\end{itemize}

Простой областью, в свою очередь, называются такие области, которые либо не превышают по размеру один пиксель, либо не являются сложными.

Алгоритм рекурсивно делит экранное пространство на области, пока не достигает простых областей, которые могут быть закрашены тем или иным цветом.

\subsection*{Трассировка лучей}

Трассировка лучей --- это более сложный и точный метод, который использует физический принцип распространения света. 

В этом методе лучи "выстреливаются" из камеры в сцену, и для каждого луча вычисляется, какие объекты он пересекает. При пересечении луча с объектом, луч может не только остановиться, но и отразиться или преломиться. 

Трассировка лучей позволяет добиться высокой степени реализма, однако она требует значительных вычислительных ресурсов.

\subsection*{Алгоритм с Z-буфером}

Алгоритм с Z-буфером (или глубинным буфером) --- это один из простейших~\cite{Rogers} методов удаления невидимых поверхностей. 

Он использует дополнительный буфер для хранения информации о глубине каждого пикселя на экране. При отрисовке каждого объекта в сцене сравнивается его глубина с уже сохраненной в Z-буфере. Если новый объект ближе к камере, его цвет записывается в цветовой буфер, а его глубина — в Z-буфер. 

Этот метод позволяет эффективно обрабатывать сложные сцены и обеспечивает высокую производительность, что позволяет использовать его и при отрисовке сцен в реальном времени.

\subsection*{Выбор подходящего алгоритма}

Было принято решение использовать алгоритм с Z-буфером. 

Алгоритм Варнока хоть и предлагает относительную простоту, он будет не совсем оптимален для отрисовки большого количества примитивов, из которых будет состоять модель загородного посёлка.

Алгоритм трассировки лучей, в свою очередь, позволяет достичь относительного реализма по сравнению с другими алгоритмами, но он делает это затрачивая большое количество вычислительных ресурсов. Поскольку целью работы не является реалистичная визуализация модели, затраты на алгоритм трассировки лучей не будут оправданными.

Алгоритм с Z-буфером, в свою очередь, предлагает и относительную скорость работы и достаточное для цели работы качество изображения.