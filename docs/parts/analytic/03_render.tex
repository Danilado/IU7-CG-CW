\section{Сравнение алгоритмов удаления невидимых поверхностей}

Алгоритмы удаления невидимых линий и поверхностей используются в машинной графике для удаления тех объектов (или частей объектов), которые перекрываются другими. Задача удаления невидимых линий и поверхностей является одной из наиболее сложных в машинной графике~\cite{Rogers}.

В этом разделе будут рассмотрены несколько алгоритмов удаления невидимых поверхностей, а для сравнения будут использованы следующие критерии:
\begin{itemize}
    \item необходимость в сортировке объектов сцены;
    \item возможность реализации оптических эффектов;
    \item временная сложность алгоритма в зависимости от: \begin{itemize}
        \item разрешения экрана;
        \item количества граней на сцене.
    \end{itemize}
\end{itemize}

\subsection*{Алгоритм Варнока}

Алгоритм Варнока (или алгоритм художника) основывается на разбиении пространства на области, которые могут быть классифицированы, как сложные и простые.

Под сложной областью подразумевается такая область (часть пространства на экране), которая превышает по размеру один пиксель и в которой возникает любой из следующих случаев:
\begin{itemize}
    \item в область попали несколько объектов;
    \item в область попал один объект, который занимает не всю область.
\end{itemize}

Простой областью, в свою очередь, называются такие области, которые либо не превышают по размеру один пиксель, либо не являются сложными.

Алгоритм рекурсивно делит экранное пространство на области, пока не достигает простых областей, которые могут быть закрашены тем или иным цветом.

Необходимости в сортировке объектов сцены для этого алгоритма нет, однако такая сортировка значительно увеличивает эффективность алгоритма~\cite{Rogers}.

Данный алгоритм не даёт возможности для реализации оптических эффектов.

Сложность алгоритма равна $O(WHN)$, где W --- ширина экрана, H --- высота экрана, N --- количество обрабатываемых граней~\cite{Rogers}.

\subsection*{Трассировка лучей}

Трассировка лучей --- это более сложный и точный метод, который использует физический принцип распространения света. 

В этом методе лучи "выстреливаются" из камеры в сцену, и для каждого луча вычисляется, какие объекты он пересекает. При пересечении луча с объектом, луч может не только остановиться, но и отразиться или преломиться. 

Необходимости в сортировке объектов сцены для этого алгоритма нет.

Трассировка лучей позволяет добиться высокой степени реализма, путём точной симуляции оптических эффектов, однако она требует значительных вычислительных ресурсов.

Сложность алгоритма равна $O(WHN)$, где W --- ширина экрана, H --- высота экрана, N --- количество обрабатываемых граней~\cite{Rogers}. Однако вычислительная сложность данного алгоритма будет значительно выше других из-за дополнительной сложности в обработке каждого трассируемого луча. Эта сложность не отображается в нотации O-большое, так как является константой.

\subsection*{Алгоритм с Z-буфером}

Алгоритм с Z-буфером (или глубинным буфером) --- это один из простейших~\cite{Rogers} методов удаления невидимых поверхностей. 

Он использует дополнительную матрицу глубин, называемую буфером для хранения информации о глубине каждого пикселя на экране. При отрисовке каждого объекта сцены, глубина пикселей этого объекта сравнивается с уже сохраненной в буфере глубиной. Если новый объект ближе к камере, то есть глубина пикселя меньше того, что записано в буфере, его цвет записывается в матрицу цветов пикселей кадра, а его глубина --- в матрицу глубин. 

Этот метод позволяет эффективно обрабатывать сложные сцены и обеспечивает высокую производительность, что позволяет использовать его и при отрисовке сцен в реальном времени.

Необходимости в сортировке объектов сцены для этого метода нет.

Данный алгоритм не даёт возможности для реализации оптических эффектов.

Сложность алгоритма равна $O(WHN)$, где W --- ширина экрана, H --- высота экрана, N --- количество обрабатываемых граней~\cite{Rogers}.

\subsection*{Вывод}

Результаты сравнения алгоритмов удаления невидимых поверхностей представлены в таблице~\ref{tbl:comparison}.

\begin{table}[h!]
    \caption{Таблица результатов сравнения алгоритмов удаления невидимых поверхностей}
    \label{tbl:comparison}
    \begin{tabular}{|l|l|l|l|}
    \hline
    Критерий                                                                             & \begin{tabular}[c]{@{}l@{}}Алгоритм\\ Варнока\end{tabular}                                                  & \begin{tabular}[c]{@{}l@{}}Алгоритм\\ Трассировки\\ Лучей\end{tabular} & \begin{tabular}[c]{@{}l@{}}Алгоритм\\ с Z-буфером\end{tabular} \\ \hline
    \begin{tabular}[c]{@{}l@{}}Необходимость\\ в сортировке граней\end{tabular}          & \begin{tabular}[c]{@{}l@{}}Нет, но сортировка \\ значительно увеличивает \\ производительность\end{tabular} & Нет                                                                    & Нет                                                            \\ \hline
    Временная сложность                                                                  & $O(WHN)$                                                                                                    & $O(WHN)$                                                               & $O(WHN)$                                                       \\ \hline
    \begin{tabular}[c]{@{}l@{}}Возможность реализации\\ оптических эффектов\end{tabular} & Нет                                                                                                         & Да                                                                     & Нет                                                            \\ \hline
    \end{tabular}
    \end{table}

Было принято решение использовать алгоритм с Z-буфером. 

Недостаток Алгоритма Варнока заключается в том, что он будет не совсем оптимален для отрисовки большого количества примитивов, из которых будет состоять модель загородного посёлка, а для улучшения производительности потребуется производить дополнительный этап --- сортировку граней, которая всё равно не исключит необходимость в повторном вычислении глубин граней.

Алгоритм трассировки лучей, в свою очередь, позволяет достичь относительного реализма по сравнению с другими алгоритмами, но он делает это затрачивая большое количество вычислительных ресурсов. Поскольку целью работы не является реалистичная визуализация модели, затраты на алгоритм трассировки лучей не будут оправданными.

Алгоритм с Z-буфером, предлагает и относительную скорость работы и достаточное для цели работы качество изображения, а также позволит повторно использовать код для вычисления карт теней, описанных в следующей секции.