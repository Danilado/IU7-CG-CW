\addcontentsline{toc}{chapter}{СПИСОК ИСПОЛЬЗОВАННЫХ ИСТОЧНИКОВ}
\begin{thebibliography}{}
  \bibitem{CGPaP} John F. Hugens. Computer Graphics Principles and Practice // 3-е издание // Бостон: Addison Wesley Professional // 1263 страницы

  \bibitem{QWFC} Heese, Raoul. Quantum Wave Function Collapse for Procedural Content Generation // статья // Institute of Electrical and Electronics Engineers (IEEE) // 13 страниц

  \bibitem{Rogers} Д. Роджерс Алгоритмические основы машинной графики // Перевод С.А. Вичес // Москва: Мир 1989г. // 512 страниц 

  \bibitem{Shadows} А.В. Мальцев Моделирование теней в 3D сценах с помощью каскадных теневых карт в режиме реального времени // статья // Журнал ИНФОРМАЦИОННЫЕ ТЕХНОЛОГИИ И ВЫЧИСЛИТЕЛЬНЫЕ СИСТЕМЫ // Москва: Российская Академия Наук 2014г. // 52 страницы

  \bibitem{LightMaps} Michael Abrash. Quake's lighting model: Surface caching. // Онлайн-ресурс: https://www.bluesnews.com/abrash/chap68.shtml // дата обращения: 03.11.2004

  % \bibitem{python3-matplotlib} Документация библиотеки Matplotlib: Visualization with Python /  [Электронный ресурс] // Matplotlib : [сайт]. — URL: https://matplotlib.org/ (дата обращения: 25.09.2024).

  % \bibitem{itmo-levenstein} Романовский И.В. Дискретный анализ: Учебное пособие для студентов, специализирующихся по прикладной маткматике и информатике. // 4-е издание, исправленное и дополненное. // СПб.: Невский диалект; БХВ-Петербург, 2008. // 336 страниц

\end{thebibliography}