\ssr{ВВЕДЕНИЕ}

\textbf{Компьютерная графика} -- это область информатики, занимающаяся созданием, обработкой и отображением изображений с использованием вычислительных технологий. Она включает в себя как двумерную, так и трёхмерную графику, а также анимацию и визуализацию данных. Актуальность компьютерной графики возрастает с развитием технологий, таких как виртуальная и дополненная реальность, которые требуют высококачественной визуализации для создания реалистичных и интерактивных пользовательских опытов. В современных приложениях, от видеоигр до медицинской визуализации, компьютерная графика играет ключевую роль в представлении информации и взаимодействии с пользователями, что делает её важной областью для исследования и развития.

Целью данного курсового проекта является разработка ПО с пользовательским интерфейсом для генерации и визуализации загородного посёлка. Сцена содержит модели домов, источник света и камеру. Интерфейс пользователя должен позволять задавать параметры для генерации посёлка: размер домов, количество домов, ширина дорог, шаблоны расположения (кварталами или случайно). 

Интерфейс приложения также должен предоставлять возможность для передвижения камеры и источника света.

Для достижения поставленной цели нужно решить следующие задачи:
\begin{itemize}
  \item сравнение существующих алгоритмов процедурной генерации сцены и алгоритмов использующихся для визуализации трёхмерной модели (сцены);
  \item выбор подходящих алгоритмов для решения поставленных задач;
  \item проектирование архитектуры и графического интерфейса ПО;
  \item выбор средств реализации ПО;
  \item разработка спроектированного ПО;
  \item замер временных характеристик разработанного ПО.
\end{itemize}
\clearpage